% arara: xelatex
\documentclass[12pt]{article}

% Time series, spring 2026

% \usepackage{physics}

\usepackage{hyperref}
\hypersetup{
    colorlinks=true,
    linkcolor=blue,
    filecolor=magenta,      
    urlcolor=cyan,
    pdftitle={Overleaf Example},
    pdfpagemode=FullScreen,
    }

\usepackage{tikzducks}

\usepackage{tikz} % картинки в tikz
\usetikzlibrary{shapes, arrows, positioning}
\usepackage{microtype} % свешивание пунктуации

\usepackage{array} % для столбцов фиксированной ширины

\usepackage{indentfirst} % отступ в первом параграфе

\usepackage{sectsty} % для центрирования названий частей
\allsectionsfont{\centering}

\usepackage{amsmath, amsfonts, amssymb} % куча стандартных математических плюшек

\usepackage{comment}

\usepackage[top=2cm, left=1.2cm, right=1.2cm, bottom=2cm]{geometry} % размер текста на странице

\usepackage{lastpage} % чтобы узнать номер последней страницы

\usepackage{enumitem} % дополнительные плюшки для списков
%  например \begin{enumerate}[resume] позволяет продолжить нумерацию в новом списке
\usepackage{caption}

\usepackage{url} % to use \url{link to web}


\newcommand{\smallduck}{\begin{tikzpicture}[scale=0.3]
    \duck[
        cape=black,
        hat=black,
        mask=black
    ]
    \end{tikzpicture}}

\usepackage{fancyhdr} % весёлые колонтитулы
\pagestyle{fancy}
\lhead{Time Series, spring 2026}
\chead{}
\rhead{Home assignments for samurai}
\lfoot{}
\cfoot{}
\rfoot{}

\renewcommand{\headrulewidth}{0.4pt}
\renewcommand{\footrulewidth}{0.4pt}

\usepackage{tcolorbox} % рамочки!

\usepackage{todonotes} % для вставки в документ заметок о том, что осталось сделать
% \todo{Здесь надо коэффициенты исправить}
% \missingfigure{Здесь будет Последний день Помпеи}
% \listoftodos - печатает все поставленные \todo'шки


% более красивые таблицы
\usepackage{booktabs}
% заповеди из докупентации:
% 1. Не используйте вертикальные линни
% 2. Не используйте двойные линии
% 3. Единицы измерения - в шапку таблицы
% 4. Не сокращайте .1 вместо 0.1
% 5. Повторяющееся значение повторяйте, а не говорите "то же"


\setcounter{MaxMatrixCols}{20}
% by crazy default pmatrix supports only 10 cols :)


\usepackage{fontspec}
\usepackage{libertine}
\usepackage{polyglossia}

\setmainlanguage{russian}
\setotherlanguages{english}

% download "Linux Libertine" fonts:
% http://www.linuxlibertine.org/index.php?id=91&L=1
% \setmainfont{Linux Libertine O} % or Helvetica, Arial, Cambria
% why do we need \newfontfamily:
% http://tex.stackexchange.com/questions/91507/
% \newfontfamily{\cyrillicfonttt}{Linux Libertine O}

\AddEnumerateCounter{\asbuk}{\russian@alph}{щ} % для списков с русскими буквами
% \setlist[enumerate, 2]{label=\asbuk*),ref=\asbuk*}

%% эконометрические сокращения
\DeclareMathOperator{\Cov}{\mathbb{C}ov}
\DeclareMathOperator{\Corr}{\mathbb{C}orr}
\DeclareMathOperator{\Var}{\mathbb{V}ar}
\DeclareMathOperator{\col}{col}
\DeclareMathOperator{\row}{row}

\let\P\relax
\DeclareMathOperator{\P}{\mathbb{P}}

\DeclareMathOperator{\E}{\mathbb{E}}
% \DeclareMathOperator{\tr}{trace}
\DeclareMathOperator{\card}{card}

\DeclareMathOperator{\Convex}{Convex}
\DeclareMathOperator{\plim}{plim}

\newcommand{\cF}{\mathcal{F}}
\newcommand{\cH}{\mathcal{H}}



\newcommand{\cN}{\mathcal{N}}
\newcommand{\RR}{\mathbb{R}}
\newcommand{\NN}{\mathbb{N}}
\newcommand{\hb}{\hat{\beta}}


\usepackage{mathtools}

\DeclarePairedDelimiter{\norm}{\lVert}{\rVert}
\DeclarePairedDelimiter{\abs}{\lvert}{\rvert}
\DeclarePairedDelimiter{\scalp}{\langle}{\rangle}
\DeclarePairedDelimiter{\ceil}{\lceil}{\rceil}



\begin{document}

You are more than welcome to openly discuss these problems. 
You don't need to hand in these problems.
The home assignments are graded only through quizzes. 
Questions with [For Fun] mark will not enter the quizzes. 

\section*{Home assignment 1}

Enters in the quizzes during the week 3: 26 January. 

\begin{enumerate}
\item   The semi-annual $y_t$ is modelled by $ETS(AAA)$ process:
    
    \[
    \begin{cases}
        u_t \sim \cN(0; 4) \\
        s_t = s_{t-2} + 0.1 u_t \\
        b_t = b_{t-1} + 0.2 u_t \\
        \ell_t = \ell_{t-1} + b_{t-1} + 0.3 u_t \\
        y_t = \ell_{t-1} + b_{t-1} + s_{t-2} + u_t \\
    \end{cases}    
    \]

    \begin{enumerate}
        \item Given that $s_{100} = 2$, $s_{99} = -1.9$, $b_{100} = 0.5$, $\ell_{100} = 4$ find 95\% prёdictive interval for $y_{102}$. 
        \item In this problem particular values of parameters are specified. 
        How many parameters are estimated in semi-annual $ETS(AAA)$ model before real forecasting?
    \end{enumerate}
\item   The $ETS(AAdN)$ model is given by the system
  \[
  \begin{cases}
  u_t  \sim \cN(0; 16) \\
  b_t = 0.9 b_{t-1} + 0.2 u_t \\
  \ell_t = \ell_{t-1} + 0.9 b_{t-1} + 0.1 u_t \\
  y_t = \ell_{t-1} + 0.9 b_{t-1} + u_t \\
  \end{cases}
  \]
  with $\ell_{100} = 20$ and $b_{100} = 2$.
\begin{enumerate}
  \item Find the 95\% predictive interval for $y_{101}$.
  \item Find conditional probability $\P(y_{102} > 30 \mid \ell_{100}, b_{100})$.
  \item Approximately find the best point forecast for $y_{10000}$.
  \item Find the 95\% predictive interval for $b_{10000}$.
\end{enumerate}
\item The semi-annual $y_t$ is modelled by $ETS(AAA)$ process:
    
    \[
    \begin{cases}
        u_t \sim \cN(0; 4) \\
        s_t = s_{t-2} + 0.1 u_t \\
        b_t = b_{t-1} + 0.2 u_t \\
        \ell_t = \ell_{t-1} + b_{t-1} + 0.3 u_t \\
        y_t = \ell_{t-1} + b_{t-1} + s_{t-2} + u_t \\
    \end{cases}    
    \]

Given that $s_{0} = 2$, $s_{-1} = -2$, $b_{0} = 0.5$, $\ell_{0} = 4$  
decompose $y_1 = 3$, $y_2 = 6$, $y_3= 4$ into trend, seasonal component and random shocks.

\end{enumerate}

\section*{Home assignment 1}

Enters in the quizzes during the week 5: 9 February.


\begin{enumerate}
\item Consider an \( MA(2) \) process defined by the equation:
\[
y_t = 5 + u_t + 2u_{t-1} + 4u_{t-2},
\]
where \((u_t)\) is a white noise process with variance \(\Var(u_t) = \sigma^2\).

    \begin{enumerate}
        \item Find the expected value \(\E(y_t)\).
        \item Find the autocorrelation function \(\rho(k) = \Corr(y_t, y_{t-k})\).
        \item Is the process \((y_t)\) stationary?
        \item Construct a 95\% confidence interval for \(y_{101}\), given \(u_{100} = 1\) and \(u_{99} = -1\).
    \end{enumerate}

\item Consider an \( MA(1) \) process defined by the equation:
\[
y_t = \mu + u_t + b u_{t-1},
\]
where \((u_t)\) is a white noise process with zero mean and variance \(\sigma^2\).

Assume we observe the following three consecutive realizations of this process:
\[
y_1 = 2, \quad y_2 = 0.5, \quad y_3 = 1.5.
\]

    \begin{enumerate}
          \item Using the method of moments, set up the system of equations by equating $\E(y_t)$, $\E(y_t^2)$ and $\E(y_t y_{t-1})$ with their sample counterparts.
          Derive estimates for the parameters $\mu$, \(\b\) and \(\sigma^2\).

        \item Based on your estimates, is the process invertible? Justify your answer.
    \end{enumerate}

\item Suppose it is known that the autocorrelation function of an MA process takes the following values:
\[
\rho(1) = -\frac{13}{19}, \quad \rho(2) = -\frac{91}{190}, \quad \rho(k) = 0 \text{ for } k \ge 3.
\]

 \begin{enumerate}
        \item What is the order of this MA process?
        \item Is there enough information to recover all parameters of the model generating this process?
        \item Recover the parameters if possible.
    \end{enumerate}

\end{enumerate}

\end{document}

